\documentclass[10pt,a4paper]{scrartcl}
\usepackage[top=1.0cm, bottom=1.0cm, left=1.0cm, right=1.0cm]{geometry}

\usepackage[utf8]{inputenc}
\usepackage[english,russian]{babel}
\usepackage{indentfirst}
\usepackage{titlesec}
\usepackage{misccorr}
	\ifx\pdfoutput\undefined
	\usepackage{graphicx}
	\else
	\usepackage[pdftex]{graphicx}
	\fi
\graphicspath{{images/}}
\usepackage{amsmath}

\titlespacing{\subsubsection}{0pt}{6pt}{0pt}

%1 страница
\begin{document}
	%\subsubsection*{Преобразования степеней}
	\begin{minipage}{0.25\linewidth}\noindent 
		\begin{align*}
		& a^{-n} = \frac {1}{a^n} \\
		& \sqrt[n]{a^m} = a ^ {\frac {m}{n}} \\
		& \sqrt[n]{-a} = -\sqrt[n]{a} \\
		& 2^{x-3} = \frac {2^3}{2^x} \\
		& a^0 = 1,\; if a \neq 0
		\end{align*}
		\begin{tabular}{lll}
		$\sqrt{2}$ & $\sqrt{3}$ & $\sqrt{5}$ \\
		1.4142                    & 1.7320                    & 2.2360                   
		\end{tabular}
	\end{minipage}
	\hfill
	\begin{minipage}{0.40\linewidth}\noindent 
		\begin{align*}
		& \text{Формула сложного радикала:} \\
		& \sqrt{a\pm \sqrt{b}} = 
			    \sqrt{\frac{
			            a + \sqrt{a^2-b}
			                }
			                {2}
			        } \pm 
			        \sqrt{\frac{
			            a - \sqrt{a^2-b}
			                }
			                {2}
			        }\\[6pt]
		& \text{Подкоренное всегда > 0: }\sqrt{ab} = \sqrt{| a |} \cdot \sqrt{|b|} \\[6pt]
		& \text{Если a\&b }  \in (0, +\infty), \; n\&m \in \mathbb {N}(2, +\infty) \text{:} \\
		& \big( \sqrt[n]{a}\big)^m = \sqrt[n]{a^m} \qquad \sqrt[m]{\sqrt[n]{a}} = \sqrt[m * n]{a}
		\end{align*}
	\end{minipage}
	\hfill
	\begin{minipage}{0.25\linewidth}\noindent
		\includegraphics[width=1\linewidth]{square_and_root}
	\end{minipage}
	
	%\subsubsection*{Полиномы}
	\begin{minipage}{0.3\linewidth}
		\begin{align*}
		& ax^2+bx+c = a(x_1 - x_1)(x-x_2) \\
		& x_1 + x_2 = \frac {-b}{a} \qquad x_1 \cdot x_2 = \frac {c}{a} \\[6pt]
		& D = b^2 -4ac \\
		& x_1,_2 = \frac{-b\pm\sqrt{D}}{2a}
		\end{align*}
	\end{minipage}
	\hfill
	\begin{minipage}{0.7\linewidth}

		\textbf{Теорема Безу для стандартного полинома:}\\
		Если a - корень, то полином F(x) без остатка делится на (x-a); \\
		\textbf{1: } (1) корень, если сумма коэфф. == 1; \\
		\textbf{2: } (-1) корень, если сумма коэфф. при четных и нечетных степенях равны; \\
		\textbf{3: } Корни полинома $\in\mathbb {Z}$, если старший коээфициент == 1; \\
					.\qquad В этом случае они -- делители свободного члена. \\
					.\qquad Значит можно найти делители св. члена, подставить, найти корень,\\
					.\qquad и разделить многочлен на (x-rdх).
	\end{minipage}

	%\subsubsection*{Логарифмы}
	\begin{minipage}{0.25\linewidth}\noindent 
		\begin{align*}
		& 2^x = 3 \; \Rightarrow \;x = \log_23 \\
		& \text{ОДЗ для }\log_ab \\
		& a>0, b>0, a \neq 1
		\end{align*}
	\end{minipage}
	\hfill
	\begin{minipage}{0.25\linewidth}\noindent 
	\begin{align*}
		& \log_ax = \frac{log_nx}{log_na} \\
		& \log{a^c}b^m = \frac{m}{c} \log_ab \\
		& \log_ab \pm \log_ac = \log_a{a \; ^{\times} \!\! / \!\! _{\div} \; b}
		\end{align*}
	\end{minipage}
	\begin{minipage}{0.5\linewidth}\noindent
		\includegraphics[width=1\linewidth]{log_and_exp}
	\end{minipage}
	
	%\subsubsection*{Тригонометрия}
	\begin{minipage}{0.40\linewidth}
	\includegraphics[width=1\linewidth]{unit_circle} \\
	\end{minipage}
	\hfill
	\begin{minipage}{0.6\linewidth}
		\begin{align*}
		& 180^\circ = 3.14 Rad \\
		& sin^2\alpha + cos^2\alpha = 1 \\
		&sin \alpha + sin \beta = 2 sin \frac{\alpha + \beta}{2} \cdot cos \frac{\alpha - \beta}{2}\\
		&sin(\alpha \pm \beta) = sin \alpha \cdot cos \beta \pm sin \beta \cdot cos \alpha \\
		&cos(\alpha \pm \beta) = sin \alpha \cdot sin \beta \pm cos \alpha \cdot cos \beta \\		
		&sin \alpha \cdot sin \beta = \frac{1}{2}(cos(\alpha - \beta) - cos(\alpha + \beta))
		\end{align*}
		\begin{minipage}{1\linewidth}
		\begin{align*}
		cos 2 \alpha = cos^2 \alpha - \sin^2 \alpha = 2cos^2x-1& \\
		=1-2sin^2&
		\end{align*}
		\end{minipage}
	\end{minipage}
	\includegraphics[width=0.59\linewidth]{arc_func}
	\hfill
	\includegraphics[width=0.39\linewidth]{sin_stratch}

\newpage
%\subsubsection*{Пределы и дифференцирование}
	\begin{minipage}{0.3\linewidth}			
		\begin{align*}
		x' = 1 \\
		c' = 0 \\
		(c \cdot u)' &= c \cdot u'\\
		(sin x)' &= cos x\\
		(cos x)' &= -sin x \\
		x'(y_0) &= \frac{1}{y'(x_0)}\\
		\\
		\\
		\end{align*}
	\end{minipage}
	\begin{minipage}{0.4\linewidth}
		\begin{align*}
		(u + v)' &= u' + v' \\
		(u \cdot v)' &= u'v + uv' \\
		g\big(f(x)\big)' &= g'\big(f(x)\big) \cdot f'(x); \quad \big(u(v)\big)' = u'(v)\cdot v'\\
		(x^n)' &= n \cdot x^{n-1} \\
		(\sqrt{x})' &= \frac {1}{2 \sqrt{x}}	\\
		\bigg(\frac{1}{x}\bigg)' &= \frac{-1}{x^2} \\
		\bigg(\frac{u}{v}\bigg)' &= \frac{u'v  - uv'}{v^2} 
		\end{align*}
	\end{minipage}
	\begin{minipage}{0.3\linewidth}
		\begin{align*}
		\\
		\\
		\\
		\\
		\\
		ln (u) &= \frac{u'}{u}\\
		\\
		ln (x) &= \frac{1}{x}\\
		\end{align*}
	\end{minipage}

\begin{figure}[h]
\begin{center}
\begin{picture}(500,300)
 \put(16,16){
  \includegraphics[width=500\unitlength]{log_and_exp_numbers}}
 \put(380,295){
  \makebox(0,0)[lb]{$\text{Касательная к графику}$}}
 \put(380,280){
  \makebox(0,0)[lb]{$y=f(a)+f'(a)(x-a)$}}
\end{picture}
\end{center}
\end{figure}
%\subsubsection*{Пределы}
\begin{minipage}{0.5\linewidth}
	\begin{align*}
	\lim_{x\to a} \big(f(x)+g(x)\big) &= \lim_{x\to a} f(x) + \lim_{x\to a} g(x) \\
	\\
	\lim_{x\to a} \big(f(x)\cdot g(x)\big) &= \lim_{x\to a} f(x) \cdot \lim_{x\to a} g(x) \\
	\\
	\lim_{x\to a} \frac{f(x)}{g(x)} &= \frac{\lim\limits_{x\to a} f(x)}{\lim\limits_{x\to a} g(x)} \\
	\lim\limits_{x\to a} f(x) = 0 \quad &\& \quad f(x) \neq 0 \quad \Rightarrow \quad  \lim\limits_{x\to a} \frac{1}{f(x)} = \infty \\
	\end{align*}
\end{minipage}
\begin{minipage}{0.5\linewidth}
	\begin{align*}
	\left.
  	\begin{array}{ccc}
  		\begin{aligned}
	\lim\limits_{x\to \infty} \Bigl(1 \,+ \,  &\frac{1}{x}\Bigr)^x &=  \\
	\\
	\lim\limits_{x\to 0} \bigl(1 \, + \, & x\bigr)^\frac{1}{x  } &= \\
	\\
	\lim\limits_{x\to 0} \bigl(1 \, + \, & a(x)\bigr)^\frac{1}{a(x)  }  &= \\
	\text{при} \thickspace & a(x)^{-e} = 0  \\
		\end{aligned}
	\end{array}
	\right\}  \scalebox{1.2}{$ \mathrm{e} $} 
	\end{align*}
\end{minipage}
%\subsubsection*{Два милиционера}
	\begin{align*}
	g_1(x) \leqslant f(x) \leqslant g_2(x) \quad &\& \quad \lim\limits_{x\to a} g_1(x) = \lim\limits_{x\to a} g_2(x) = l  \quad \Rightarrow \quad \lim\limits_{x\to a} f(x) =l \\
	\end{align*}
%\subsubsection*{Теоремы в практике}
	\begin{align*}
	\text{Произведение ограниченной функиции на бесконечно малую} \quad \Rightarrow \quad \lim  & \rightarrow 0 \\
	\lim\limits_{x\to \infty} \frac{sinx}{x} & = 0
	\end{align*}
\end{document}